\documentclass[12pt,a4paper]{article}

\usepackage[utf8]{inputenc}
\usepackage[T1]{fontenc}
\usepackage{mathtools}
\usepackage{amsmath}
\usepackage{amssymb}
\usepackage{amsthm}
\usepackage{geometry}
\usepackage{hyperref}
\usepackage{booktabs}
\usepackage{array}
\usepackage{xcolor}
\usepackage{float}
\usepackage{afterpage}
\usepackage{cleveref}

% cleveref names for custom theorem environments
\crefname{theorem}{Theorem}{Theorems}
\crefname{lemma}{Lemma}{Lemmas}
\crefname{proposition}{Proposition}{Propositions}
\crefname{corollary}{Corollary}{Corollaries}
\crefname{definition}{Definition}{Definitions}
\crefname{remark}{Remark}{Remarks}
\crefname{example}{Example}{Examples}
\crefname{equation}{Eq.}{Eqs.}
\crefname{table}{Table}{Tables}
\crefname{section}{Section}{Sections}

\geometry{a4paper, margin=2.5cm}

\hypersetup{
  colorlinks=true,
  linkcolor=blue!60!black,
  citecolor=blue!60!black,
  urlcolor=blue!60!black,
  pdftitle={Differential-Geometric Analysis of the Conical Helix},
  pdfauthor={Dogan Balban},
  pdfkeywords={conical helix, Frenet-Serret, curvature, torsion, arc length, Archimedean spiral, Basel problem}
}

\theoremstyle{plain}
\newtheorem{theorem}{Theorem}[section]
\newtheorem{lemma}[theorem]{Lemma}
\newtheorem{proposition}[theorem]{Proposition}
\newtheorem{corollary}[theorem]{Corollary}

\theoremstyle{definition}
\newtheorem{definition}[theorem]{Definition}
\newtheorem{example}{Example}[section]

\theoremstyle{remark}
\newtheorem{remark}{Remark}[section]

\newcommand{\R}{\mathbb{R}}
\newcommand{\N}{\mathbb{N}}
\newcommand{\abs}[1]{\left|#1\right|}
\newcommand{\norm}[1]{\left\|#1\right\|}
\newcommand{\vek}[1]{\boldsymbol{#1}}


\title{\textbf{Differential-Geometric Analysis of the Conical Helix}\\[0.5em]
\large A Complete Treatment of Curvature, Torsion,\\
Frenet Frame, Arc Length, and Geometric Properties}
\author{Dogan Balban}
\date{\today}

\begin{document}

\maketitle

\begin{abstract}
This paper presents a complete differential-geometric analysis of the conical helix defined by the parametrization
\begin{equation*}
\gamma(\theta) =
\begin{pmatrix}
\dfrac{6}{\pi^2}\theta \cos(\theta) \\[8pt]
\dfrac{6}{\pi^2}\theta \sin(\theta) \\[8pt]
-\dfrac{3}{4} + \dfrac{3}{\pi}\,\theta
\end{pmatrix}, \qquad \theta \geq 0.
\end{equation*}
The curve combines the properties of an Archimedean spiral (in the $xy$-projection) with a linear increase in height, and lies entirely on the cone $z = -\frac{3}{4} + \frac{\pi}{2}r$. We compute in closed form: the tangent vector, principal normal, binormal (Frenet frame), curvature, torsion, arc length, pitch angle, and the asymptotic behaviour of all these quantities. A key result is that the spiral constant $a = 6/\pi^2 = 1/\zeta(2)$ connects number theory and differential geometry directly.
\end{abstract}

\tableofcontents
\newpage

%=============================================================
\section{Introduction and Background}
%=============================================================

\subsection{The Conical Helix}

A \emph{helix} is a space curve that winds spirally around an axis~\cite{docarmo,pressley}. In the classical case of the \emph{cylindrical helix}, the radius remains constant and both curvature and torsion are constant. The curve studied here is a fundamental generalisation: the radius grows linearly with the angle (Archimedean spiral), while the height also increases linearly. The curve therefore winds on a \emph{cone} and is called a \textbf{conical helix}.

\begin{definition}[Conical Helix]\label{def:helix}
The conical helix is the space curve $\gamma : [0, \infty) \to \R^3$ defined by:
\begin{equation}\label{eq:gamma}
\boxed{
\gamma(\theta) =
\begin{pmatrix}
\dfrac{6}{\pi^2}\,\theta \cos(\theta) \\[10pt]
\dfrac{6}{\pi^2}\,\theta \sin(\theta) \\[10pt]
-\dfrac{3}{4} + \dfrac{3}{\pi}\,\theta
\end{pmatrix}
}
\end{equation}
with structural constants
\begin{equation}\label{eq:constants}
a := \frac{6}{\pi^2} \approx 0.607927101854027, \qquad
b := \frac{3}{\pi} \approx 0.954929658551372.
\end{equation}
\end{definition}

\begin{remark}[Structural Constants and the Basel Problem]\label{rem:constants}
The spiral constant $a = 6/\pi^2$ is the reciprocal of the famous Euler sum $\sum_{n=1}^\infty \frac{1}{n^2} = \frac{\pi^2}{6}$, known as the Basel problem (L.~Euler, 1734/1740)~\cite{euler1735,hardy}. The ratio of the two constants is remarkable:
\begin{equation}\label{eq:ratio}
\frac{b}{a} = \frac{3/\pi}{6/\pi^2} = \frac{\pi}{2}.
\end{equation}
As shown in \cref{sec:kegel}, this ratio equals exactly the slope of the cone.
\end{remark}

\subsection{Component Form}

In component form, \cref{eq:gamma} reads:
\begin{align}
x(\theta) &= a\theta\cos(\theta), \label{eq:x}\\
y(\theta) &= a\theta\sin(\theta), \label{eq:y}\\
z(\theta) &= -\tfrac{3}{4} + b\theta. \label{eq:z}
\end{align}

The $xy$-projection of $\gamma$ is the \emph{Archimedean spiral} $r(\theta) = a\theta$~\cite{archimedes,lawrence,spirale}, while $z(\theta)$ depends linearly on $\theta$.

\subsection{Regularity}

\begin{proposition}[Regularity]\label{prop:regular}
The curve $\gamma$ is regular for all $\theta > 0$, i.e.\ $\dot{\gamma}(\theta) \neq \vek{0}$.
\end{proposition}

\begin{proof}
We have $\dot{z}(\theta) = b = 3/\pi > 0$ for all $\theta$. Therefore the third component of the tangent vector is always non-zero, which implies $\dot{\gamma} \neq \vek{0}$.
\end{proof}

%=============================================================
\section{Position on the Cone}\label{sec:kegel}
%=============================================================

\begin{theorem}[Cone Equation]\label{thm:cone}
The conical helix $\gamma$ lies entirely on the cone
\begin{equation}\label{eq:cone}
\mathcal{K} := \left\{(x,y,z) \in \R^3 \;\middle|\; z = -\frac{3}{4} + \frac{\pi}{2}\sqrt{x^2+y^2}\right\}.
\end{equation}
\end{theorem}

\begin{proof}
The distance of a curve point from the $z$-axis is:
\[
r(\theta) = \sqrt{x(\theta)^2 + y(\theta)^2} = \sqrt{a^2\theta^2\cos^2\theta + a^2\theta^2\sin^2\theta} = a\theta.
\]
Substituting into the right-hand side of \cref{eq:cone}:
\[
-\frac{3}{4} + \frac{\pi}{2} \cdot a\theta = -\frac{3}{4} + \frac{\pi}{2} \cdot \frac{6}{\pi^2}\theta = -\frac{3}{4} + \frac{3}{\pi}\theta = z(\theta).
\]
\end{proof}

\begin{corollary}[Cone Slope]
The slope of the cone generator is $\pi/2$; the corresponding cone angle from the horizontal ($xy$-plane) is:
\begin{equation}
\psi_{\mathcal{K}} = \arctan\!\left(\frac{\pi}{2}\right) \approx 57.518^\circ.
\end{equation}
\end{corollary}

\begin{remark}[Two Cone Angles]\label{rem:coneangles}
Two different angles are used in the literature to characterise a cone, depending on convention:
\begin{itemize}
\item The \emph{elevation angle from the horizontal} (from the $xy$-plane): $\psi_{\mathcal{K}} = \arctan(\pi/2) \approx 57.52^\circ$.
\item The \emph{opening half-angle from the rotation axis} (from the $z$-axis): $\alpha = \arctan(2/\pi) \approx 32.48^\circ$.
\end{itemize}
Both angles are complementary: $\psi_{\mathcal{K}} + \alpha = 90^\circ$.
\end{remark}

%=============================================================
\section{Tangent Vector and Arc Length}
%=============================================================

\subsection{Tangent Vector}

\begin{proposition}[Tangent Vector]\label{prop:tangent}
The tangent vector $\dot{\gamma}(\theta)$ is:
\begin{equation}\label{eq:tangent}
\dot{\gamma}(\theta) =
\begin{pmatrix}
a(\cos\theta - \theta\sin\theta)\\
a(\sin\theta + \theta\cos\theta)\\
b
\end{pmatrix}.
\end{equation}
\end{proposition}

\begin{proposition}[Speed]\label{prop:speed}
\begin{equation}\label{eq:speed}
\abs{\dot{\gamma}(\theta)} = \sqrt{a^2(1+\theta^2) + b^2}.
\end{equation}
\end{proposition}

\begin{proof}
\begin{align*}
\abs{\dot{\gamma}}^2 &= a^2(\cos\theta - \theta\sin\theta)^2 + a^2(\sin\theta + \theta\cos\theta)^2 + b^2 \\
&= a^2\left[\cos^2\theta - 2\theta\sin\theta\cos\theta + \theta^2\sin^2\theta\right]\\
&\quad + a^2\left[\sin^2\theta + 2\theta\sin\theta\cos\theta + \theta^2\cos^2\theta\right] + b^2\\
&= a^2\left[\underbrace{\cos^2\theta + \sin^2\theta}_{=1} + \theta^2\underbrace{(\sin^2\theta + \cos^2\theta)}_{=1}\right] + b^2\\
&= a^2(1 + \theta^2) + b^2.
\end{align*}
\end{proof}

\begin{corollary}[Initial Speed]\label{cor:speed0}
Bei $\theta = 0$ gilt:
\[
\abs{\dot{\gamma}(0)} = \sqrt{a^2 + b^2} = \sqrt{\frac{36}{\pi^4} + \frac{9}{\pi^2}} = \frac{3}{\pi}\sqrt{\frac{4}{\pi^2}+1} \approx 1.13202.
\]
\end{corollary}

\begin{corollary}[Asymptotic Speed]\label{cor:speed_asymp}
For $\theta \to \infty$:
\[
\abs{\dot{\gamma}(\theta)} \sim a\theta = \frac{6}{\pi^2}\theta.
\]
\end{corollary}

\subsection{Second and Third Derivatives}

\begin{proposition}[Second Derivative]\label{prop:gamma2}
\begin{equation}\label{eq:gamma2}
\ddot{\gamma}(\theta) =
\begin{pmatrix}
a(-2\sin\theta - \theta\cos\theta)\\
a(2\cos\theta - \theta\sin\theta)\\
0
\end{pmatrix}.
\end{equation}
\end{proposition}

\begin{proposition}[Third Derivative]\label{prop:gamma3}
\begin{equation}\label{eq:gamma3}
\dddot{\gamma}(\theta) =
\begin{pmatrix}
a(-3\cos\theta + \theta\sin\theta)\\
a(-3\sin\theta - \theta\cos\theta)\\
0
\end{pmatrix}.
\end{equation}
\end{proposition}

\subsection{Arc Length}

\begin{theorem}[Arc Length]\label{thm:arc}
The arc length of the conical helix from $\theta = 0$ to $\theta = T$ is:
\begin{equation}\label{eq:arc}
L(T) = \frac{a}{2}\left[T\sqrt{T^2 + c^2} + c^2\,\operatorname{arcsinh}\!\left(\frac{T}{c}\right)\right],
\end{equation}
where
\begin{equation}\label{eq:c}
c^2 := \frac{a^2 + b^2}{a^2} = 1 + \frac{\pi^2}{4} \approx 3.4674, \qquad c \approx 1.8621.
\end{equation}
\end{theorem}

\begin{proof}
From \cref{prop:speed}:
\[
L(T) = \int_0^T \sqrt{a^2(1+t^2)+b^2}\,dt = a\int_0^T \sqrt{t^2 + \frac{a^2+b^2}{a^2}}\,dt = a\int_0^T\sqrt{t^2+c^2}\,dt.
\]
Das Integral $\int\sqrt{t^2+c^2}\,dt = \frac{1}{2}\left[t\sqrt{t^2+c^2}+c^2\,\operatorname{arcsinh}(t/c)\right]$ is a standard integral~\cite{abramowitz,dlmf}.
\end{proof}

\begin{remark}[The Constant $c$]
The constant $c = \sqrt{1+\pi^2/4}$ follows directly from the ratio $b/a = \pi/2$ (cf.~\cref{rem:constants}). It governs the long-run growth rate of the arc length.
\end{remark}

\begin{proposition}[Asymptotic Arc Length]\label{prop:arc_asymp}
For $T \to \infty$:
\begin{equation}
L(T) \sim \frac{a}{2}T^2 = \frac{3}{\pi^2}T^2.
\end{equation}
\end{proposition}

\begin{proof}
For $T \gg c$: $\sqrt{T^2+c^2} \approx T$ and $\operatorname{arcsinh}(T/c) \approx \ln(2T/c) = o(T^2)$, so the dominant term is $\frac{a}{2}T^2$.
\end{proof}

\begin{example}[Arc Length per Revolution]\label{ex:arc}
Arc length of the first five revolutions:

\begin{table}[H]
\renewcommand{\arraystretch}{1.6}
\centering
\caption{Arc length of the conical helix per revolution}
\label{tab:arc}
\begin{tabular}{cccc}
\toprule
\textbf{Revolution} $n$ & $\theta$: from & $\theta$: to & \textbf{Arc length} $L$ \\
\midrule
1 & $0$ & $2\pi$ & $14.5507$ \\
2 & $2\pi$ & $4\pi$ & $36.7221$ \\
3 & $4\pi$ & $6\pi$ & $60.4258$ \\
4 & $6\pi$ & $8\pi$ & $84.3867$ \\
5 & $8\pi$ & $10\pi$ & $108.3747$ \\
\bottomrule
\end{tabular}
\end{table}

Cumulative arc length after $n$ full revolutions:

\begin{table}[H]
\renewcommand{\arraystretch}{1.6}
\centering
\caption{Cumulative arc length after $n$ revolutions}
\label{tab:arc_total}
\begin{tabular}{cccc}
\toprule
\textbf{Revolutions} $n$ & $\theta_{\max}$ & $r(\theta_{\max})$ & \textbf{Arc length} $L$ \\
\midrule
1  & $2\pi$   & $3.820$  & $14.5507$ \\
2  & $4\pi$   & $7.639$  & $51.2728$ \\
3  & $6\pi$   & $11.459$ & $111.6985$ \\
5  & $10\pi$  & $19.099$ & $304.2361$ \\
10 & $20\pi$  & $38.197$ & $1204.9663$ \\
\bottomrule
\end{tabular}
\end{table}
\end{example}

%=============================================================
\section{The Frenet--Serret Frame}
%=============================================================

\subsection{Overview}

The \emph{Frenet--Serret frame} (also called the moving trihedron, cf.~\cite{docarmo,kreyszig,pressley}) is an orthonormal coordinate system that moves along the curve and completely describes its local geometry. Er besteht aus drei Vektoren:
\begin{itemize}
\item $\vek{T}(\theta)$ -- unit tangent vector (direction of motion),
\item $\vek{N}(\theta)$ -- principal normal unit vector (direction of curvature),
\item $\vek{B}(\theta)$ -- binormal unit vector ($\vek{B} = \vek{T}\times\vek{N}$).
\end{itemize}

\subsection{Unit Tangent Vector}

\begin{definition}[Unit Tangent Vector]\label{def:T}
\begin{equation}\label{eq:T}
\vek{T}(\theta) = \frac{\dot{\gamma}(\theta)}{\abs{\dot{\gamma}(\theta)}} = \frac{1}{\sqrt{a^2(1+\theta^2)+b^2}}
\begin{pmatrix}
a(\cos\theta - \theta\sin\theta)\\
a(\sin\theta + \theta\cos\theta)\\
b
\end{pmatrix}.
\end{equation}
\end{definition}

\begin{proposition}[Pitch Angle]\label{prop:pitch}
The angle $\psi(\theta)$ between $\vek{T}$ and the $xy$-plane (pitch angle) is:
\begin{equation}\label{eq:pitch}
\psi(\theta) = \arctan\!\left(\frac{b}{a\sqrt{1+\theta^2}}\right) = \arctan\!\left(\frac{\pi}{2\sqrt{1+\theta^2}}\right).
\end{equation}
\end{proposition}

\begin{remark}
At $\theta \to 0$: $\psi(0) = \arctan(\pi/2) \approx 57.52^\circ$, exactly the cone angle from \cref{sec:kegel}. As $\theta \to \infty$, $\psi \to 0^\circ$: the curve becomes increasingly flat.
\end{remark}

\begin{table}[H]
\renewcommand{\arraystretch}{1.8}
\centering
\caption{Pitch angle $\psi(\theta)$ of the conical helix}
\label{tab:pitch}
\begin{tabular}{ccc}
\toprule
$\theta$ & $\psi(\theta)$ (rad) & $\psi(\theta)$ (degrees) \\
\midrule
$\theta \to 0$ & $1.0039$ & $57.517^\circ$ \\
$\pi/2$        & $0.7007$ & $40.150^\circ$ \\
$\pi$          & $0.4446$ & $25.475^\circ$ \\
$2\pi$         & $0.2421$ & $13.869^\circ$ \\
$4\pi$         & $0.1240$ & $7.103^\circ$ \\
$10\pi$        & $0.0499$ & $2.861^\circ$ \\
\bottomrule
\end{tabular}
\end{table}

\subsection{The Cross Product $\dot{\gamma}\times\ddot{\gamma}$}

\begin{lemma}[Cross Product]\label{lem:cross}
The cross product $\dot{\gamma}\times\ddot{\gamma}$ has components:
\begin{align}
(\dot{\gamma}\times\ddot{\gamma})_x &= -ab(2\cos\theta - \theta\sin\theta), \label{eq:cx}\\
(\dot{\gamma}\times\ddot{\gamma})_y &= -ab(2\sin\theta + \theta\cos\theta), \label{eq:cy}\\
(\dot{\gamma}\times\ddot{\gamma})_z &= a^2(2+\theta^2). \label{eq:cz}
\end{align}
\end{lemma}

\begin{proof}
Using $\dot{\gamma}$ from \cref{eq:tangent} and $\ddot{\gamma}$ from \cref{eq:gamma2}:
\begin{align*}
(\dot{\gamma}\times\ddot{\gamma})_z &= \dot{x}\ddot{y} - \dot{y}\ddot{x}\\
&= a(\cos\theta-\theta\sin\theta)\cdot a(2\cos\theta-\theta\sin\theta)\\
&\quad - a(\sin\theta+\theta\cos\theta)\cdot a(-2\sin\theta-\theta\cos\theta)\\
&= a^2\bigl[\cos^2\theta(2-\theta^2) + \sin^2\theta(2-\theta^2) + 2\theta^2\bigr]\\
&= a^2(2-\theta^2+2\theta^2) = a^2(2+\theta^2). \qedhere
\end{align*}
\end{proof}

\begin{proposition}[Norm of the Cross Product]\label{prop:cross_norm}
\begin{equation}\label{eq:cross_norm}
\abs{\dot{\gamma}\times\ddot{\gamma}}^2 = a^2b^2(4+\theta^2) + a^4(2+\theta^2)^2.
\end{equation}
\end{proposition}

\begin{proof}
Inserting \cref{eq:cx,eq:cy,eq:cz}:
\begin{align*}
\abs{\dot{\gamma}\times\ddot{\gamma}}^2 &= a^2b^2(2\cos\theta-\theta\sin\theta)^2 + a^2b^2(2\sin\theta+\theta\cos\theta)^2 + a^4(2+\theta^2)^2\\
&= a^2b^2\bigl[(2\cos\theta-\theta\sin\theta)^2+(2\sin\theta+\theta\cos\theta)^2\bigr] + a^4(2+\theta^2)^2\\
&= a^2b^2(4+\theta^2) + a^4(2+\theta^2)^2. \qedhere
\end{align*}
\end{proof}

%=============================================================
\section{Curvature}
%=============================================================

\begin{theorem}[Curvature of the Conical Helix]\label{thm:curvature}
The curvature $\kappa(\theta)$ of the conical helix is:
\begin{equation}\label{eq:curvature}
\kappa(\theta) = \frac{\sqrt{a^2b^2(4+\theta^2)+a^4(2+\theta^2)^2}}{\left[a^2(1+\theta^2)+b^2\right]^{3/2}}.
\end{equation}
Using $b = \frac{\pi}{2}a$, this simplifies to:
\begin{equation}\label{eq:curvature2}
\kappa(\theta) = \frac{a\sqrt{\frac{\pi^2}{4}(4+\theta^2)+(2+\theta^2)^2}}{\left[a^2\left(1+\theta^2+\frac{\pi^2}{4}\right)\right]^{3/2}}.
\end{equation}
\end{theorem}

\begin{proof}
Follows from the Frenet formula $\kappa = \abs{\dot{\gamma}\times\ddot{\gamma}}/\abs{\dot{\gamma}}^3$ together with \cref{prop:cross_norm,prop:speed}.
\end{proof}

\begin{proposition}[Initial Curvature]\label{prop:kappa0}
Bei $\theta = 0$ gilt:
\begin{equation}
\kappa(0) = \frac{2a\sqrt{a^2+b^2}}{(a^2+b^2)^{3/2}} = \frac{2a}{a^2+b^2}
= \frac{4\pi^2}{3(\pi^2+4)} \approx 0.948799.
\end{equation}
\end{proposition}

\begin{proof}
Setting $\theta = 0$ in \cref{eq:curvature}, the numerator becomes:
\[
\sqrt{a^2b^2(4+0)+a^4(2+0)^2} = \sqrt{4a^2b^2+4a^4} = 2a\sqrt{b^2+a^2}.
\]
The denominator is $(a^2 \cdot 1 + b^2)^{3/2} = (a^2+b^2)^{3/2}$. Hence:
\[
\kappa(0) = \frac{2a\sqrt{a^2+b^2}}{(a^2+b^2)^{3/2}} = \frac{2a}{a^2+b^2}. \qedhere
\]
\end{proof}

\begin{proposition}[Asymptotic Curvature]\label{prop:kappa_asymp}
For $\theta \to \infty$:
\begin{equation}
\kappa(\theta) \sim \frac{a}{\theta} = \frac{6}{\pi^2\theta}.
\end{equation}
\end{proposition}

\begin{proof}
For large $\theta$: numerator $\sim a^2\theta^2$, denominator $\sim a^3\theta^3$, so $\kappa \sim 1/(a\theta)$.
\end{proof}

\begin{table}[H]
\renewcommand{\arraystretch}{1.8}
\centering
\caption{Curvature and radius of curvature of the conical helix}
\label{tab:curvature}
\begin{tabular}{cccc}
\toprule
$\theta$ & $\kappa(\theta)$ & $\varrho = 1/\kappa$ & $\abs{\dot{\gamma}(\theta)}$ \\
\midrule
$0$      & $0.948799$ & $1.054$  & $1.1320$ \\
$\pi/2$  & $0.681835$ & $1.467$  & $1.4810$ \\
$\pi$    & $0.446906$ & $2.238$  & $2.2201$ \\
$2\pi$   & $0.249876$ & $4.002$  & $3.9839$ \\
$4\pi$   & $0.129307$ & $7.734$  & $7.7229$ \\
$10\pi$  & $0.052256$ & $19.137$ & $19.132$ \\
\bottomrule
\end{tabular}
\end{table}

\begin{remark}[Comparison with the Cylindrical Helix]
For a \emph{cylindrical helix} with radius $R$ and pitch $h$, both curvature $\kappa = R/(R^2+h^2)$ and torsion $\tau = h/(R^2+h^2)$ are \emph{constant}~\cite{docarmo,pressley}. For the conical helix both decrease monotonically ($\kappa \sim 1/\theta$, $\tau \sim 1/\theta^2$) -- a fundamental geometric generalisation~\cite{gray,struik,izumiya}.
\end{remark}

%=============================================================
\section{Torsion}
%=============================================================

\begin{theorem}[Torsion of the Conical Helix]\label{thm:torsion}
The torsion $\tau(\theta)$ of the conical helix is:
\begin{equation}\label{eq:torsion}
\tau(\theta) = \frac{a^2b(6+\theta^2)}{a^2b^2(4+\theta^2)+a^4(2+\theta^2)^2} = \frac{b(6+\theta^2)}{b^2(4+\theta^2)+a^2(2+\theta^2)^2}.
\end{equation}
\end{theorem}

\begin{proof}
Using the formula $\tau = (\dot{\gamma}\times\ddot{\gamma})\cdot\dddot{\gamma}/\abs{\dot{\gamma}\times\ddot{\gamma}}^2$.

\textbf{Numerator.} Using $\dddot{\gamma}$ from \cref{eq:gamma3} and the cross product from \cref{lem:cross}:
\begin{align*}
(\dot{\gamma}\times\ddot{\gamma})\cdot\dddot{\gamma}
&= -ab(2\cos\theta-\theta\sin\theta)\cdot a(-3\cos\theta+\theta\sin\theta)\\
&\quad + (-ab)(2\sin\theta+\theta\cos\theta)\cdot a(-3\sin\theta-\theta\cos\theta) + 0\\
&= a^2b\bigl[(2\cos\theta-\theta\sin\theta)(3\cos\theta-\theta\sin\theta)\\
&\qquad +(2\sin\theta+\theta\cos\theta)(3\sin\theta+\theta\cos\theta)\bigr]\\
&= a^2b\bigl[6\cos^2\theta + 6\sin^2\theta + \theta^2\cos^2\theta + \theta^2\sin^2\theta\bigr]\\
&= a^2b(6+\theta^2).
\end{align*}
The denominator is $\abs{\dot{\gamma}\times\ddot{\gamma}}^2$ from \cref{prop:cross_norm}.
\end{proof}

\begin{proposition}[Asymptotic Torsion]\label{prop:torsion_asymp}
For $\theta \to \infty$:
\begin{equation}
\tau(\theta) \sim \frac{b}{a^2\theta^2} = \frac{3/\pi}{(6/\pi^2)^2\theta^2} = \frac{\pi^3}{12\theta^2}.
\end{equation}
\end{proposition}

\begin{table}[H]
\renewcommand{\arraystretch}{1.8}
\centering
\caption{Torsion of the conical helix}
\label{tab:torsion}
\begin{tabular}{cc}
\toprule
$\theta$ & $\tau(\theta)$ \\
\midrule
$\theta\to 0$ & $1.11774$ \\
$\pi/2$       & $0.60917$ \\
$\pi$         & $0.23417$ \\
$2\pi$        & $0.06429$ \\
$4\pi$        & $0.01631$ \\
$10\pi$       & $0.00262$ \\
\bottomrule
\end{tabular}
\end{table}

\begin{proposition}[Ratio $\kappa/\tau$]\label{prop:kappa_tau}
The ratio of curvature and torsion satisfies:
\begin{equation}
\frac{\kappa(\theta)}{\tau(\theta)} = \frac{\sqrt{a^2b^2(4+\theta^2)+a^4(2+\theta^2)^2}}{a^2b(6+\theta^2)/(a^2b^2(4+\theta^2)+a^4(2+\theta^2)^2)^{1/2}} \cdot \frac{1}{a^2b^2(4+\theta^2)+a^4(2+\theta^2)^2}.
\end{equation}
For $\theta \to \infty$:
\begin{equation}
\frac{\kappa(\theta)}{\tau(\theta)} \sim \frac{a\theta}{b} = \frac{2\theta}{\pi}.
\end{equation}
\end{proposition}

%=============================================================
\section{Principal Normal and Binormal Vectors}
%=============================================================

\subsection{Principal Normal Vector}

\begin{definition}[Principal Normal Vector]\label{def:N}
The principal unit normal vector is:
\begin{equation}\label{eq:N}
\vek{N}(\theta) = \frac{1}{\kappa(\theta)\abs{\dot{\gamma}(\theta)}^2}
\left(\ddot{\gamma}(\theta) - \frac{\dot{\gamma}(\theta)\cdot\ddot{\gamma}(\theta)}{\abs{\dot{\gamma}(\theta)}^2}\dot{\gamma}(\theta)\right).
\end{equation}
\end{definition}

\begin{lemma}[Inner Product $\dot{\gamma}\cdot\ddot{\gamma}$]\label{lem:dotdot}
\begin{equation}
\dot{\gamma}\cdot\ddot{\gamma} = a^2\theta.
\end{equation}
\end{lemma}

\begin{proof}
\begin{align*}
\dot{\gamma}\cdot\ddot{\gamma}
&= a(\cos\theta-\theta\sin\theta)\cdot a(-2\sin\theta-\theta\cos\theta)\\
&\quad + a(\sin\theta+\theta\cos\theta)\cdot a(2\cos\theta-\theta\sin\theta) + b\cdot 0.
\end{align*}
Expanding the two products individually:
\begin{align*}
&a^2(\cos\theta-\theta\sin\theta)(-2\sin\theta-\theta\cos\theta)\\
&\quad = a^2\bigl[-2\sin\theta\cos\theta - \theta\cos^2\theta
+ 2\theta\sin^2\theta + \theta^2\sin\theta\cos\theta\bigr],\\[6pt]
&a^2(\sin\theta+\theta\cos\theta)(2\cos\theta-\theta\sin\theta)\\
&\quad = a^2\bigl[2\sin\theta\cos\theta - \theta\sin^2\theta
+ 2\theta\cos^2\theta - \theta^2\sin\theta\cos\theta\bigr].
\end{align*}
On addition the mixed terms $\pm 2\sin\theta\cos\theta$ and
$\pm\theta^2\sin\theta\cos\theta$ cancel:
\begin{align*}
\dot{\gamma}\cdot\ddot{\gamma}
&= a^2\bigl[\theta(\cos^2\theta+\sin^2\theta)\bigr] = a^2\theta. \qedhere
\end{align*}
\end{proof}

\subsection{Binormal Vector}

\begin{definition}[Binormal Vector]\label{def:B}
The unit binormal vector is:
\begin{equation}\label{eq:B}
\vek{B}(\theta) = \vek{T}(\theta) \times \vek{N}(\theta) = \frac{\dot{\gamma}\times\ddot{\gamma}}{\abs{\dot{\gamma}\times\ddot{\gamma}}}.
\end{equation}
\end{definition}

\begin{example}[Frenet Frame at $\theta = 2\pi$]\label{ex:frenet}
At $\theta = 2\pi$, numerically:
\begin{align*}
\vek{T}(2\pi) &= \begin{pmatrix}0{,}15259\\0{,}95878\\0{,}23970\end{pmatrix}, &
\vek{N}(2\pi) &= \begin{pmatrix}-0{,}98555\\0{,}16566\\-0{,}03523\end{pmatrix}, &
\vek{B}(2\pi) &= \begin{pmatrix}-0{,}07348\\-0{,}23086\\0{,}97021\end{pmatrix}.
\end{align*}
One verifies: $\abs{\vek{T}} = \abs{\vek{N}} = \abs{\vek{B}} = 1$ and $\vek{T}\perp\vek{N}$, $\vek{T}\perp\vek{B}$, $\vek{N}\perp\vek{B}$.
\end{example}

%=============================================================
\section{The Frenet--Serret Equations}
%=============================================================

\begin{theorem}[Frenet--Serret Equations~{\cite{frenet,serret}}]\label{thm:frenet}
With respect to the natural parametrization (arc length $s$):
\begin{equation}\label{eq:frenet}
\frac{d\vek{T}}{ds} = \kappa\,\vek{N}, \qquad
\frac{d\vek{N}}{ds} = -\kappa\,\vek{T} + \tau\,\vek{B}, \qquad
\frac{d\vek{B}}{ds} = -\tau\,\vek{N}.
\end{equation}
\end{theorem}

\begin{remark}
The Frenet--Serret equations describe how the trihedron $(\vek{T},\vek{N},\vek{B})$ rotates along the curve. Curvature $\kappa$ governs rotation in the osculating plane; torsion $\tau$ governs the twisting about the tangent axis. By Lancret's theorem~\cite{lancret,struik}, a space curve is a (generalised) helix if and only if $\kappa/\tau = \mathrm{const}$. Since for the conical helix $\kappa/\tau \sim 2\theta/\pi \to \infty$, it is not a helix in this classical sense, but a geometrically richer generalisation.
\end{remark}

%=============================================================
\section{Osculating, Normal, and Rectifying Planes}
%=============================================================

\begin{definition}[Planes of the Frenet Frame]\label{def:planes}
At every regular point $\gamma(\theta)$, the Frenet frame defines three distinguished planes:
\begin{itemize}
\item \textbf{Osculating plane}: spanned by $\vek{T}$ and $\vek{N}$; contains the tangent and the principal normal.
\item \textbf{Normal plane}: spanned by $\vek{N}$ and $\vek{B}$; perpendicular to the tangent.
\item \textbf{Rectifying plane}: spanned by $\vek{T}$ and $\vek{B}$; contains the tangent and the binormal.
\end{itemize}
\end{definition}

\begin{proposition}[Normal Plane Equation]\label{prop:normalplane}
The normal plane at the point $\gamma(\theta_0)$ has the equation:
\begin{equation}
\vek{T}(\theta_0)\cdot\bigl((x,y,z) - \gamma(\theta_0)\bigr) = 0.
\end{equation}
\end{proposition}

%=============================================================
\section{Asymptotic Behaviour}
%=============================================================

\begin{theorem}[Asymptotic Behaviour]\label{thm:asymp}
As $\theta \to \infty$, the following asymptotic approximations hold:
\begin{align}
\abs{\dot{\gamma}(\theta)} &\sim a\theta, \label{eq:asymp_speed}\\
L(\theta) &\sim \frac{a}{2}\theta^2, \label{eq:asymp_arc}\\
\kappa(\theta) &\sim \frac{1}{a\theta} = \frac{\pi^2}{6\theta}, \label{eq:asymp_kappa}\\
\tau(\theta) &\sim \frac{b}{a^2\theta^2} = \frac{\pi^3}{12\theta^2}, \label{eq:asymp_tau}\\
\psi(\theta) &\sim \frac{b}{a\theta} = \frac{\pi}{2\theta}. \label{eq:asymp_pitch}
\end{align}
\end{theorem}

\begin{corollary}[Asymptotic Ratios]\label{cor:ratios}
For $\theta \to \infty$:
\begin{equation}
\frac{\kappa(\theta)}{\tau(\theta)} \sim \frac{2\theta}{\pi}, \qquad
\frac{\kappa(\theta)}{\psi(\theta)} \sim \frac{2}{\pi^2} \cdot \frac{1}{\theta}.
\end{equation}
\end{corollary}

\begin{remark}[Geometric Interpretation]
Asymptotically the curve approaches a flat spiral in the $xy$-plane since $\psi(\theta) \to 0$ and $\kappa(\theta) \to 0$. At the same time, the radius $r = a\theta \to \infty$: the curve moves unboundedly away from the origin.
\end{remark}

%=============================================================
\section{Numerical Curve Points}
%=============================================================

\begin{table}[H]
\renewcommand{\arraystretch}{1.6}
\centering
\caption{Selected points on the conical helix}
\label{tab:points}
\begin{tabular}{ccccccc}
\toprule
$\theta$ & $x(\theta)$ & $y(\theta)$ & $z(\theta)$ & $r(\theta)$ & $\kappa(\theta)$ & $\tau(\theta)$ \\
\midrule
$0$    & $0$        & $0$        & $-0.750$ & $0$        & $0.9488$ & $1.1177$ \\
$\pi$  & $-1{,}905$ & $0$        & $2.250$  & $1.905$  & $0.4469$ & $0.2342$ \\
$2\pi$ & $3.810$  & $0$        & $5.250$  & $3.810$  & $0{,}2499$ & $0.0643$ \\
$3\pi$ & $-5{,}715$ & $0$        & $8.250$  & $5.715$  & $0.1680$ & $0.0286$ \\
$4\pi$ & $7.620$  & $0$        & $11.250$ & $7.620$  & $0.1293$ & $0.0163$ \\
$5\pi$ & $-9{,}524$ & $0$        & $14.250$ & $9.524$  & $0.1037$ & $0.0104$ \\
\bottomrule
\end{tabular}
\end{table}

%=============================================================
\section{Summary of Formulae}
%=============================================================

\begin{table}[H]
\renewcommand{\arraystretch}{2.0}
\centering
\caption{Formula Summary Part~1: Parametrization, Derivatives, and Arc Length}
\label{tab:summary1}
\begin{tabular}{lp{8cm}}
\toprule
\textbf{Quantity} & \textbf{Formula} \\
\midrule
\multicolumn{2}{l}{\textit{Parametrization}} \\[2pt]
Constants & $a = \dfrac{6}{\pi^2} = \dfrac{1}{\zeta(2)}$, \quad $b = \dfrac{3}{\pi}$, \quad $\dfrac{b}{a} = \dfrac{\pi}{2}$ \\[6pt]
Curve $\gamma(\theta)$ & $\left(a\theta\cos\theta,\; a\theta\sin\theta,\; -\tfrac{3}{4}+b\theta\right)$ \\[6pt]
Cone equation & $z = -\dfrac{3}{4} + \dfrac{\pi}{2}r$ \\[6pt]
\midrule
\multicolumn{2}{l}{\textit{Derivatives}} \\[2pt]
$\dot{\gamma}(\theta)$ & $\left(a(\cos\theta-\theta\sin\theta),\; a(\sin\theta+\theta\cos\theta),\; b\right)$ \\[6pt]
$\ddot{\gamma}(\theta)$ & $\left(a(-2\sin\theta-\theta\cos\theta),\; a(2\cos\theta-\theta\sin\theta),\; 0\right)$ \\[6pt]
$\abs{\dot{\gamma}(\theta)}$ & $\sqrt{a^2(1+\theta^2)+b^2}$ \\[6pt]
\midrule
\multicolumn{2}{l}{\textit{Arc Length}} \\[2pt]
$L(T)$ & $\dfrac{a}{2}\!\left[T\sqrt{T^2+c^2}+c^2\operatorname{arcsinh}\!\left(\dfrac{T}{c}\right)\right]$ \\[6pt]
$c^2$ & $1+\dfrac{\pi^2}{4} \approx 3{,}467$ \\[6pt]
Asymptotic & $L(T) \sim \dfrac{a}{2}T^2$ \\[6pt]
\bottomrule
\end{tabular}
\end{table}

\begin{table}[H]
\renewcommand{\arraystretch}{2.0}
\centering
\caption{Formula Summary Part~2: Frenet Quantities and Identities}
\label{tab:summary2}
\begin{tabular}{lp{8cm}}
\toprule
\textbf{Quantity} & \textbf{Formula} \\
\midrule
\multicolumn{2}{l}{\textit{Frenet Quantities}} \\[2pt]
Curvature $\kappa(\theta)$ & $\dfrac{\sqrt{a^2b^2(4+\theta^2)+a^4(2+\theta^2)^2}}{\left[a^2(1+\theta^2)+b^2\right]^{3/2}}$ \\[10pt]
Asymptotic curvature & $\kappa \sim \dfrac{\pi^2}{6\theta}$ \\[6pt]
Torsion $\tau(\theta)$ & $\dfrac{b(6+\theta^2)}{b^2(4+\theta^2)+a^2(2+\theta^2)^2}$ \\[10pt]
Asymptotic torsion & $\tau \sim \dfrac{\pi^3}{12\theta^2}$ \\[6pt]
Pitch angle & $\psi = \arctan\!\left(\dfrac{\pi}{2\sqrt{1+\theta^2}}\right)$ \\[6pt]
\midrule
\multicolumn{2}{l}{\textit{Identities}} \\[2pt]
& $\dot{\gamma}\cdot\ddot{\gamma} = a^2\theta$ \\[4pt]
& $(\dot{\gamma}\times\ddot{\gamma})\cdot\dddot{\gamma} = a^2b(6+\theta^2)$ \\[4pt]
& $\kappa/\tau \sim 2\theta/\pi$ \\[4pt]
\bottomrule
\end{tabular}
\end{table}

%=============================================================
\section{Conclusion}
%=============================================================

This paper has presented a complete differential-geometric analysis of the conical helix $\gamma(\theta) = \left(a\theta\cos\theta,\, a\theta\sin\theta,\, -\frac{3}{4}+b\theta\right)^T$ präsentiert. The main results are:

\begin{enumerate}
\item The curve lies on the cone $z = -\frac{3}{4}+\frac{\pi}{2}r$ (cf.~\cref{thm:cone}), because $b/a = \pi/2$ provides exactly the cone slope.

\item The tangent vector and arc length are given in closed form; the arc length grows asymptotically as $\theta^2$ (cf.~\cref{thm:arc,prop:arc_asymp}).

\item Curvature $\kappa\sim 1/\theta$ and torsion $\tau\sim 1/\theta^2$ both decrease monotonically, in contrast to the cylindrical helix with constant values (cf.~\cref{thm:curvature,thm:torsion}). The non-constant ratio $\kappa/\tau \sim 2\theta/\pi$ confirms by Lancret's theorem that the conical helix is not a classical helix.

\item The triple product $(\dot{\gamma}\times\ddot{\gamma})\cdot\dddot{\gamma} = a^2b(6+\theta^2)$ yields an elegant closed form for the torsion (cf.~\cref{thm:torsion}).

\item The Frenet frame is completely determined; the Frenet--Serret equations (\cref{thm:frenet}) describe its evolution along the curve.

\item The spiral constant $a = 6/\pi^2 = 1/\zeta(2)$, the reciprocal of the Euler sum, gives the curve a deep number-theoretic connection (cf.~\cref{rem:constants}).
\end{enumerate}

\begin{thebibliography}{99}

% Own works
\bibitem{spirale}
D.~Balban,
\textit{Differential-Geometric Analysis of the Archimedean Spiral
$r(\theta) = \frac{6}{\pi^2}\theta$: Curvature, Arc Length, Equidistant
Parametrization, and Connection to the Basel Problem},
Independent mathematical publication, 2025.

\bibitem{kegel}
D.~Balban,
\textit{Mathematical Analysis of the Cone
$z = -\frac{3}{4}+\frac{\pi}{2}r$: Parametrization, Surface Area,
Volume, Curvature, and Developable Surfaces},
Independent mathematical publication, 2025.

\bibitem{balban2026helix}
D.~Balban,
\textit{From the Quadratic Sequence to the Conical Helix: Geometric
Transformation of Diophantine Structures and Embedding of Primes
in Three-Dimensional Space},
Independent mathematical publication, 2026.

% Classical differential geometry
\bibitem{docarmo}
M.~P.~do~Carmo,
\textit{Differential Geometry of Curves and Surfaces}, revised~ed.
Dover Publications, Mineola, NY, 2016.

\bibitem{pressley}
A.~Pressley,
\textit{Elementary Differential Geometry}, 2nd~ed.
Springer, London, 2010.

\bibitem{struik}
D.~J.~Struik,
\textit{Lectures on Classical Differential Geometry}, 2nd~ed.
Dover Publications, New York, 1988.

\bibitem{gray}
A.~Gray, E.~Abbena, and S.~Salamon,
\textit{Modern Differential Geometry of Curves and Surfaces with Mathematica}, 3rd~ed.
Chapman \& Hall/CRC, Boca Raton, FL, 2006.

\bibitem{kreyszig}
E.~Kreyszig,
\textit{Differential Geometry}.
Dover Publications, New York, 1991.

% Frenet-Serret: historical
\bibitem{frenet}
J.~F.~Frenet,
\textit{Sur les courbes à double courbure},
Doctoral thesis, Toulouse, 1847; also in:
\textit{Journal de Mathématiques Pures et Appliquées}, Vol.~17, pp.~437--447, 1852.

\bibitem{serret}
J.~A.~Serret,
\textit{Sur quelques formules relatives à la théorie des courbes à double courbure},
\textit{Journal de Mathématiques Pures et Appliquées}, Vol.~16, pp.~193--207, 1851.

\bibitem{lancret}
M.~A.~Lancret,
\textit{Mémoire sur les courbes à double courbure},
\textit{Mémoires présentés à l'Institut par divers savants}, Vol.~1, pp.~416--454, 1806.

% Standard integrals
\bibitem{abramowitz}
M.~Abramowitz and I.~A.~Stegun (eds.),
\textit{Handbook of Mathematical Functions}, 10th~ed.
Dover Publications, New York, 1992.

\bibitem{dlmf}
F.~W.~J.~Olver et~al.\ (eds.),
\textit{NIST Digital Library of Mathematical Functions},
Release~1.2.1, 2024, \url{https://dlmf.nist.gov}.

% Archimedean spiral
\bibitem{lawrence}
J.~D.~Lawrence,
\textit{A Catalog of Special Plane Curves}.
Dover Publications, New York, 1972.

\bibitem{archimedes}
Archimedes of Syracuse,
\textit{On Spirals} (\textit{De Spiralibus}), ca.~225~\textsc{bc};
English translation: T.~L.~Heath,
\textit{The Works of Archimedes},
Cambridge University Press, 1897; Dover reprint, 2002.

% Basel problem
\bibitem{euler1735}
L.~Euler,
\textit{De summis serierum reciprocarum},
\textit{Commentarii Academiae Scientiarum Petropolitanae}, Vol.~7,
pp.~123--134, 1740 (presented 1735).

\bibitem{hardy}
G.~H.~Hardy and E.~M.~Wright,
\textit{An Introduction to the Theory of Numbers}, 6th~ed.
Oxford University Press, Oxford, 2008.

% Conical helices
\bibitem{izumiya}
S.~Izumiya and N.~Takeuchi,
\textit{New Special Curves and Developable Surfaces},
\textit{Turkish Journal of Mathematics}, Vol.~28, No.~2, pp.~153--163, 2004.

\bibitem{ali}
A.~T.~Ali,
\textit{Position Vectors of Curves in the Galilean Space $G_3$},
\textit{Matematički Vesnik}, Vol.~64, No.~3, pp.~200--210, 2012.

\end{thebibliography}

\end{document}
